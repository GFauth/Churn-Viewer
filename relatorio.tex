\documentclass[12pt, a4paper]{article}

% --- Pacotes Fundamentais ---
\usepackage[utf8]{inputenc}     % Codificação de caracteres
\usepackage[T1]{fontenc}        % Codificação de fontes
\usepackage[brazil]{babel}      % Idioma para hifenização e títulos
\usepackage{amsmath, amssymb}   % Pacotes matemáticos
\usepackage{geometry}           % Configuração de margens
\usepackage{hyperref}           % Links clicáveis
\usepackage{indentfirst}        % Indenta o primeiro parágrafo das seções

\geometry{
 a4paper,
 left=3cm,
 right=2cm,
 top=3cm,
 bottom=2cm
}

\title{\textbf{Relatório: Implementação de Sistema de Indexação e Pesquisa de Dados}}
\author{Gabriel Iza Fauth \\ Melchior Boaretto Neto}
\date{\today}

\begin{document}

\maketitle

\begin{abstract}
Este relatório descreve a implementação de um sistema de gerenciamento de dados 
utilizando indexação via Árvore B em memória principal e persistência em arquivo binário. 
O sistema foi desenvolvido para gerenciar um conjunto de dados de cancelamento de clientes (Churn), 
permitindo operações de inserção, busca eficiente por chave primária e filtragem sequencial por atributos secundários.
\end{abstract}

\section{Dos dados utilizados}

O conjunto de dados selecionado para este projeto 
é o \textit{Telco Customer Churn}.

\subsection{Origem}
Os dados foram obtidos originalmente da plataforma Kaggle e 
representam o perfil de consumo e status de contrato de clientes. 
O arquivo encontra-se em formato CSV.

\subsection{Esquema de Dados Adotado}
Foram selecionados os seguintes atributos, mapeados para uma 
estrutura de tamanho fixo de 83 bytes (antes de eventual padding):

\begin{itemize}
    \item \textbf{ID do Cliente (15 bytes):} Identificador único.
    \item \textbf{Gênero (10 bytes):} Male ou Female.
    \item \textbf{Cancelou (5 bytes):} Indicador de \textit{Churn} (Yes/No).
    \item \textbf{Contrato (40 bytes):} Tipo de contrato (Mensal, Anual, etc.).
    \item \textbf{Valor Mensal (float - 4 bytes):} Custo mensal.
    \item \textbf{Meses (int - 4 bytes):} Tempo de permanência.
    \item \textbf{Idade (int - 4 bytes):} Idade do cliente.
\end{itemize}

\section{Índices e Estruturas de Dados}

Toda a eficiência do projeto baseia-se na implementação de uma estrutura de dados hierárquica
para indexação.

\subsection{Estrutura de Indexação: Árvore B}
Foi implementada uma \textbf{Árvore B} com grau mínimo $t=3$. Esta escolha
justifica-se pelo fato das operações serem feitas assintoticamente em $O(\log_t n)$.
Além disso, diferentemente da classe das \textit{ABPs}, a altura da árvore se mantém
extremamente curta.

\begin{itemize}
    \item \textbf{Atributo Indexado (Chave):} O campo \texttt{id\_cliente} é a chave primária de indexação.
    \item \textbf{Implementação:} A árvore reside em memória principal durante a execução. 
    Cada nodo pode conter entre $t-1$ e $2t-1$ chaves. O split de nodos ocorre na descida durante a inserção.
\end{itemize}

\subsection{Persistência de Dados}
A persistência utiliza manipulação direta de arquivos binários.
\begin{itemize}
    \item O sistema converte os objetos da classe \texttt{Cliente} para \textit{bytes} utilizando a biblioteca 
    \texttt{struct} do Python (Serialização).
    \item Isso permite acesso aleatório e leitura sequencial rápida, 
    eliminando o \textit{overhead} de \textit{parsing} de texto exigido por 
    arquivos CSV a cada execução.
\end{itemize}

\section{Implementação e Interface}

\subsection{Tecnologias Utilizadas}
\begin{itemize}
    \item \textbf{Linguagem:} Python.
    \item \textbf{Repositório:} \url{https://github.com/GFauth/Churn-Viewer}
\end{itemize}

\subsection{Interface do Usuário}
A interação ocorre via Interface de Linha de Comando (CLI). O sistema apresenta um menu interativo 
com estas funcionalidades:
\begin{enumerate}
    \item \textbf{Buscar Cliente por ID:} Utiliza o algoritmo de busca da Árvore B.
    \item \textbf{Filtragem Avançada:} Permite filtrar registros por atributos não 
    indexados (Gênero, Contrato, Valores, etc...) percorrendo a árvore.
    \item \textbf{Estatísticas:} Cálculo de médias de valores mensais segregados por status de cancelamento.
    \item \textbf{Gerenciamento de Arquivo:} Opção para resetar a base binária a partir do CSV original.
\end{enumerate}


\subsection{Performance de Busca por ID vs Filtragem}
A busca por ID comportou-se conforme a expectativa teórica, 
retornando resultados instantaneamente.

As operações de filtragem (e.g. buscar todos os clientes do sexo masculino) tem desempenho inferior. 
Como apenas o ID é indexado, filtros quaisquer exigem cobrir toda a árvore, resultando em complexidade linear.
Percebe-se, pois, que se a base de dados fosse da ordem de 100 ou 1000 vezes mais massiva, far-se-ia necessário
o uso de \textit{Árvores B+} e/ou arquivos invertidos.

\section{Conclusão}
O projeto cumpriu os requisitos de manipulação de arquivos e implementação de estruturas de dados. 
Para a base de dados do projeto, a utilização da Árvore B foi eficaz para a pesquisa de registros 
por chave primária. 

\end{document}